\section{Введение}

Задача машинного обучения с учителем предполагает выбор предсказательной модели из некоторого параметрического семейства. Обычно такой выбор связан с некоторыми статистическими гипотезами, например, максимизацией некоторого функционала качества. 
\begin{definition}
    Модель прогнозирования, которая соответствует этим статистическим гипотезам, называется \textbf{адекватной} моделью.
\end{definition}

При проведении эксперимента зачастую дана конечная обучающая выборка.

\begin{definition}
    Размер выборки, необходимый для построения адекватной модели прогнозирования, называется \textbf{достаточным}.
\end{definition}

В работе \cite{Grabovoy2022} рассматриваются различные методы оценки объема выборки в обобщенных линейных моделях, включая статистические, эвристические и байесовские методы. Анализируются такие методы, как тест на множители Лагранжа, тест на отношение правдоподобия, статистика Вальда, кросс-валидация, бутстрап, критерий Куллбэка-Лейблера, критерий средней апостериорной дисперсии, критерий среднего охвата, критерий средней длины и максимизация полезности. Авторы статьи указывают на возможное развитие темы, которое заключается в поиске метода, сочетающего байесовский и статистический подходы для оценки размера выборки для недостаточного доступного размера выборки.

В \cite{MOTRENKO2014743} рассматривается новый метод определения размера выборки в логистической регрессии. Метод использует кросс-валидацию и дивергенцию Кульбака-Лейблера между апостериорными распределениями параметров модели на схожих подвыборках.