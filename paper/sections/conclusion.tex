\section*{Заключение}                         % Заголовок
\addcontentsline{toc}{section}{Заключение}    % Добавляем его в оглавление

Предложены подходы к определению достаточного размера выборки, основанные на значениях функции правдоподобия на бустрапированных подвыборках и близости апостериорных распределений параметров модели на схожих подвыборках. Первые два позволяют определять достаточный размер выборки на любом датасете с задачей регрессии или классификации. Доказана корректность предложенных подходов при определенных ограничениях на используемую модель, а также предложен метод прогнозирования функции правдоподобия в случае недостаточного размера выборки. Доказана теорема о моментах предельного апостериорного распределения параметров в модели линейной регрессии. Проведенный вычислительный эксперимент позволяет анализировать свойства предложенных методов и их эффективность. Определено параметрическое семейство функций, аппроксимирующее функцию ошибки для набора датасетов. 